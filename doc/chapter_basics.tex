\section{Podstawowe informacje dotyczące przetwarzania obrazów}
\subsection{Podstawowe definicje}
\subsubsection{Piksel}
Piksel jest najmniejszym elementem obrazu cyfrowego. Piksel może określać
\begin{itemize}
\item odcień szarości
\item kolor, wtedy \(f(x, y) = [a(x, y), b(x, y),...]\)\\
  gdzie $a, b,...$ to natężenie poszczególnych barw
\item wskaźnik na element tablicy barw
\end{itemize}
\subsubsection{Obraz cyfrowy}
Obraz cyfrowy jest macierzą pikseli, postaci
\begin{gather*}
  f = f(x, y), x = 0,1,2,...,N-1; y = 0,1,2,...,M-1
\end{gather*}
gdzie
\begin{itemize}
\item \(f(x, y)\) - pojedynelement macierzy, piksel
\item \(M, N\) - szerokość oraz wysokość obrazu
\end{itemize}
\subsubsection{Obraz binarny}
Obraz binarny jest to obraz cyfrowy którego składowe piksele przyjmują wartości
\begin{gather*}
  f(x, y) = z, z \in {0, 1}
\end{gather*}.
