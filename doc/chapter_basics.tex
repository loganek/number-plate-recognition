\section{Podstawowe informacje dotyczące przetwarzania obrazów}
\subsection{Podstawowe definicje}
\subsubsection{Piksel}
Piksel jest najmniejszym elementem obrazu cyfrowego. Piksel może określać
\begin{itemize}
\item odcień szarości
\item kolor, wtedy \(f(x, y) = [a(x, y), b(x, y),...]\)\\
  gdzie $a, b,...$ to natężenie poszczególnych barw
\item wskaźnik na element tablicy barw
\end{itemize}
\subsubsection{Obraz cyfrowy}
Obraz cyfrowy definiowany jest jako funkcja:
\begin{gather*}
  f = f(x, y), x = 0,1,2,...,N-1; y = 0,1,2,...,M-1
\end{gather*}
gdzie
\begin{itemize}
\item \(f(x, y)\) - pojedynczy element macierzy, piksel
\item \(M, N\) - szerokość oraz wysokość obrazu
\end{itemize}
Zbiorem wartości dla funkcji obrazu są piksele, dla których każda składowa spełnia warunek:
\begin{gather*}
  f(x, y) \in N
\end{gather*}, gdzie:
\begin{itemize}
\item N - Zbiór liczb naturalnych z przedziału $<0; 2^b>$. Wartość \textit{b} określa liczbę bitów potrzebnych do reprezentacji pojedynczej składowej piksela w obrazie.
\end{itemize}
Przykład obrazu cyfrowego znajduje się na rysunku~\ref{fig:basics_image_color}.
\subsubsection{Obraz cyfrowy jednokanałowy}
Obraz cyfrowy jednokanałowy jest to obraz składający się z pikseli zawierających tylko jedną składową. Zwykle obrazy jednokanałowe wykorzystywane są w celu reprezentacji obrazu w skali szarości. Rysunek~\ref{fig:basics_image_gray} przedstawia jednokanałowy obraz cyfrowy, powstały w wyniku uśrednienia składowych obrazu trójkanałowego.
\subsubsection{Obraz binarny}
Obraz binarny jest to obraz cyfrowy jednokanałowy, którego piksele przyjmują wartości
\begin{gather*}
  f(x, y) = z, z \in \{0, 1\}
\end{gather*}.
Obraz binarny przedstawiony został na rysunku~\ref{fig:basics_image_binary}.
\begin{figure}
  \centering
  \begin{subfigure}[b]{0.45\textwidth}
    \includegraphics[width=\textwidth]{img/basics-image-color}
    \caption{Trójkanałowy obraz cyfrowy(kolorowy)}
    \label{fig:basics_image_color}
  \end{subfigure}
  ~
  \begin{subfigure}[b]{0.45\textwidth}
    \includegraphics[width=\textwidth]{img/basics-image-gray}
    \caption{Jednokanałowy obraz cyfrowy(skala szarości)}
    \label{fig:basics_image_gray}
  \end{subfigure}
  ~
  \begin{subfigure}[b]{0.45\textwidth}
    \includegraphics[width=\textwidth]{img/basics-image-binary}
    \caption{Obraz jednokanałowy(binarny)}
    \label{fig:basics_image_binary}
  \end{subfigure}
  \caption{Rodzaje obrazów cyfrowych}\label{fig:image_examples}
\end{figure}
\subsection{Podstawowe algorytmy przetwarzania obrazów}
\subsubsection{Kadrowanie obrazu}
Operacja kadrowania polega na pozostawieniu na obrazie wejściowym tylko tych pikseli, które znajdują się w zadanym, prostokątnym obszarze:
\begin{gather*}
  \forall p \in I\quad p.x \in {p1.x, p2.x} \wedge p.y \in {p1.y, p2.y}
\end{gather*}, gdzie:
\begin{itemize}
  \item I - obraz wejściowy
  \item p1 - współrzędne lewego górnego rogu obszaru kadru,
  \item p2 - współrzędne prawego dolnego rogu obszaru kadru
\end{itemize}. Na rysunku ~\ref{fig:crop_image} przedstawiona została operacja kadrowania obrazu. Celem kadrowania jest usunięcie z obrazu informacji, które są zbędne podczas wykonywania analizy obrazu, ponieważ mogą negatywnie wpłynąć na czas oraz rezultaty wykonania algorytmów.
\begin{figure}
  \centering
  \begin{subfigure}[b]{0.45\textwidth}
    \includegraphics[width=\textwidth]{img/crop-image-before}
    \caption{Obraz wejściowy dla operacji kadrowania}
    \label{fig:crop_image_before}
  \end{subfigure}
  ~
  \begin{subfigure}[b]{0.45\textwidth}
    \includegraphics[width=\textwidth]{img/crop-image-after}
    \caption{Obraz wyjściowy, poddany operacji kadrowania}
    \label{fig:crop_image_after}
  \end{subfigure}
  \caption{Operacja kadrowania obrazu}\label{fig:crop_image}
\end{figure}
\subsubsection{Operacje morfologiczne}
Operacje morfologiczne na obrazach wykorzystywane są do filtracji morfologicznej oraz analizy kształtów obiektów na obrazie. Algorytmy morfologiczne są podstawą dla wielu bardziej złożonych algorytmów analizy wizji komputerowej.
\paragraph {Dylatacja obrazu} to rozszerzenie obrazu wykorzystując zadany element strukturalny. Operacja definiowana jest wzorem
\begin{gather*}
  A \oplus B \equiv \bigcup \limits_{b \in B} A_b
\end{gather*}.
Przykładowy wynik operacji dylatacji przedstawiony jest na rysunku~\ref{fig:dilate}.
\begin{figure}
  \centering
  \includegraphics[width=15cm]{img/dilate}
  \caption{Obraz binarny poddany procesowi dylatacji z zadanym elementem strukturalnym}
  \label{fig:dilate}
\end{figure}
\paragraph {Erozja obrazu} to usuwanie zwężanie obrazu z wykorzystaniem zadanego elementu strukturalnego. Operacja definiowana jest wzorem
\begin{gather*}
  A \ominus B \equiv \bigcap \limits_{b \in B} A_{-b}
\end{gather*}.
Przykładowy wynik operacji erozji obrazu przedstawiony został na rysunku~\ref{fig:erode}.
\begin{figure}
  \centering
  \includegraphics[width=15cm]{img/erode}
  \caption{Obraz binarny poddany procesowi erozji z zadanym elementem strukturalnym}
  \label{fig:erode}
\end{figure}
\paragraph {Zamknięcie i otwarcie obrazu} to złożenie dwóch wyżej wymienionych operacji (dylatacji oraz erozji) w odpowiedni sposób.
\begin{itemize}
\item zamknięcie obrazu definiowane jest w następujący sposób:
  \begin{gather*}
    A \bullet B = (A \oplus B) \ominus B
  \end{gather*}, czyli najpierw wykonywana jest operacja dylatacji obrazu, a następnie przekształcony obraz poddawany jest operacji erozji
\item otwarcie obrazu definiowane jest wzorem:
  \begin{gather*}
    A \circ B = (A \ominus B) \oplus B
  \end{gather*}. W przypadku algorytmu otwarcia obrazu, w pierwszej kolejności wykonywana jest operacja erozji, a następnie operacja dylatacji obrazu.
\end{itemize}
Na rysunku~\ref{fig:open_close} przedstawiony został wynik działania obydwu algorytmów na tym samym obrazie, z tym samym elementem strukturalnym.
\begin{figure}
  \centering
  \includegraphics[width=15cm]{img/open-close}
  \caption{Obraz binarny poddany operacji otwarcia oraz zamknięcia obrazu, dla takiego samego elementu strukturalnego}
  \label{fig:open_close}
\end{figure}
\subsubsection{Progowanie} \label{sssec:threshold}
Progowanie polega na podziale pikseli obrazu na dwie grupy, poprzez wybranie określonej wartości progowej $t$. Każdy piksel jest porównywany z wartością progową, i w zależności od tego, czy wartość piksela jest większa od wartości progowej, czy mniejsza, w tej samej pozycji nowo powstałego obrazu, przypisuje się wartość $1$, lub $0$. Operację można opisać wzorem:
\begin{gather*}
  I_{out}(x, y) = \left\{\begin{matrix}
  1, dla \: I_{in}(x, y) > t\\
  0, dla \: I_{in}(x, y) \leq t
  \end{matrix}\right. x=0,1,2,...,N-1; y=0,1,2,...,M-1
\end{gather*},
gdzie:
\begin{itemize}
\item $I_{in}$ - obraz wejściowy
\item $I_{out}$ - obraz wyjściowy
\item $N$ - szerokość obrazu
\item $M$ - wysokość obrazu
\end{itemize}
Na rysunku ~\ref{fig:threshold_image} przedstawiony został wynik operacji progowania na przykładowym obrazie monochromatycznym.
\begin{figure}
  \centering
  \begin{subfigure}[b]{0.45\textwidth}
    \includegraphics[width=\textwidth]{img/threshold-before}
    \caption{Obraz wejściowy dla operacji progowania}
    \label{fig:threshold_before}
  \end{subfigure}
  ~
  \begin{subfigure}[b]{0.45\textwidth}
    \includegraphics[width=\textwidth]{img/threshold-after}
    \caption{Obraz wyjściowy, poddany operacji progowania}
    \label{fig:threshold_after}
  \end{subfigure}
  \caption{Operacja progowania z wartością progu $t=128$ dla obrazu jednokanałowego}\label{fig:threshold_image}
\end{figure}

\paragraph{Progowanie obrazów z większą ilością składowych}\mbox{}\\
Zazwyczaj operacji progowania poddawane są obrazy jednokanałowe. Istnieje jednak możliwość wykonania operacji progowania dla obrazów wielokanałowych (np. dla obrazów RGB). Operacja ta definiowana jest następującym wzorem:
\begin{gather*}
  I_{out}(x, y) = \left\{\begin{matrix}
  1, \; \text{jeżeli} \; \forall c \in C, \; I_{in}(x, y, c) > t(c)\\
  0, \; \text{jeżeli} \; \exists c \in C, \; I_{in}(x, y, c) \leq t(c)
  \end{matrix}\right.\\ x=0,1,2,...,N-1; y=0,1,2,...,M-1; c=0,1,2,...,C
\end{gather*},
gdzie $C$ to liczba kanałów w obrazie.
Algorytm progowania operujący na wielu kanałach może zostać wykorzystany, kiedy znana jest dokładna barwa badanego obiektu.

\subsubsection{Rozmycie obrazu}
Algorytmy rozmycia stosuje się w celu usunięcia z obrazu szumów oraz zbędnych detali. Rozmycie polega na zastosowaniu dla każdego piksela w obrazie funkcji przyjmującej jako argument dany piksel wraz z jego najbliższym sąsiedztwem (tzw. jądro). Powszechnie stosuje się następujące funkcje rozmycia:
\begin{itemize}
\item mediana,
\item średnia,
\item rozmycie Gaussa
\end{itemize}
. Jądro zastosowane w filtrze może mieć dowolne kształty i rozmiary, natomiast w przypadku dużych jąder, algorytm będzie wykonywał się znacznie dłużej, a obraz może zostać rozmyty zbyt bardzo (utracone zostaną istotne informacje zapisane w wejściowym obrazie).
Rysunek~\ref{fig:lena_smooth} przedstawia zaszumiony obraz oraz wynik działania algorytmu rozmycia medianowego.
\begin{figure}
  \centering
  \includegraphics[width=15cm]{img/lena-smooth}
  \caption{Wynik działania filtra rozmycia medianowego z jądrem w kształcie kwadratu 3x3}
  \label{fig:lena_smooth}
\end{figure}

\subsection{Operacje na histogramach}
Bardzo często podczas analizy obrazu pod kątem znajdowania segmentów, wykorzystywana jest analiza histogramu. W celu ułatwienia analizy, a także poprawy jej wyników, przed jej rozpoczęciem można zastosować niżej wymienione algorytmy.
\subsubsection{Wyrównanie histogramu}
Metoda wyrównywania histogramu ma na celu zmianę kontrastu obrazu. Algorytm wykorzystywany jest w przypadku, gdy zarówno tło, jak i pierwszy plan obrazu, są ciemne lub jasne. Wadą metody wyrównywania histogramu jest możliwe wzmocnienie zakłóceń występujących na obrazie, gdyż algorytm traktuje je jak sygnał opisujący prawidłowy obraz. Przed zastosowaniem tej metody, warto zatem zastosować jeden z algorytmów rozmycia obrazu, opisywany we wcześniejszej części tego rozdziału.\\
Metoda wyrównania histogramu sprowadza się do wyznaczenia tablicy LUT(\textit{ang. Lookup table}), na podstawie której wyznaczone zostaną wartości poszczególnych pikseli obrazu wejściowego. W celu wyznaczenia tablicy LUT, należy wyznaczyć dystrybuantę rozkładu prawdopodobieństwa dla wartości pikseli obrazu:
\begin{gather*}
  D(i) = \sum\limits_{j=0}^i p(j)
\end{gather*}, gdzie i jest wartością występującą na obrazie, a p(j) jest prawdopodobieństwem wystąpienia wartości j w obrazie. Wykorzystując wyznaczone wartości dystrybuanty, możemy opisać tablicę LUT za pomocą wzoru:
\begin{gather*}
  LUT(i) = \frac{D(i)-D_0}{1-D_0}*K
\end{gather*}, gdzie i to wartość składowej obrazu wejściowego, $D_0$ to pierwsza wartość dystrybuanty różna od zera, a K jest maksymalną wartością występującym w obrazie wejściowym.\\
Rysunek~\ref{fig:equalizehistogram} przedstawia efekt działania operacji wyrównania histogramu na przykładowym obrazie.
\begin{figure}
  \centering
  \begin{subfigure}[b]{0.45\textwidth}
    \includegraphics[width=\textwidth]{img/equalize-histogram-before}
    \caption{Obraz wejściowy dla operacji wyrównania histogramu}
    \label{fig:equalize_histogram_before}
  \end{subfigure}
  ~
  \begin{subfigure}[b]{0.45\textwidth}
    \includegraphics[width=\textwidth]{img/equalize-histogram-after}
    \caption{Obraz wyjściowy, poddany operacji wyrównania histogramu}
    \label{fig:equalize_histogram_after}
  \end{subfigure}
  ~
  \begin{subfigure}[b]{0.45\textwidth}
    \includegraphics[width=\textwidth]{img/equalize-histogram-after}
    \caption{Histogram obrazu wejściowego}
%    \label{fig:equalize_histogram_histogram_before}
  \end{subfigure}
  ~
  \begin{subfigure}[b]{0.45\textwidth}
    \includegraphics[width=\textwidth]{img/equalize-histogram-after}
    \caption{Histogram obrazu wyjściowego}
    \label{fig:equalize_histogram_histogram_after}
  \end{subfigure}
 % \caption{Operacja wyrównania histogramu dla obrazu jednokanałowego}\label{fig:equalize_histogram}
\end{figure}

\subsubsection{Wygładzenie histogramu}
Często histogram obrazu zawiera w sobie nagłe skoki wartości, które utrudniają analizę histogramu, gdyż mogą reprezentować fałszywe ekstrema. Wygładzanie histogramu ma na celu usunięcie takich skoków, zachowując przy tym oryginalny kształt histogramu. Dla każdej wartości histogramu wejściowego \textit{$H_{in}$} należy zastosować następującą operację:
\begin{gather*}
  \vee i \in <H_{in}(0), H_{in}(H_{max})> H_{out}(i) = H_{in}(i-1) + H_{in}(i) + H_{in}(i+1)
\end{gather*},
gdzie \textit{$H_{out}$} jest histogramem wyjściowym, a $H_{max}$ największym indeksem w histogramie. Przykład działania algorytmu wygładzania histogramu przedstawiony został na rysunku~\ref{fig:histogram_smooth}
\begin{figure}
  \centering
  \begin{subfigure}[b]{0.45\textwidth}
    \includegraphics[width=\textwidth]{img/smooth-histogram-before}
    \caption{Obraz wejściowy dla operacji wygładzenia histogramu}
    \label{fig:equalize_histogram_before}
  \end{subfigure}
  ~
  \begin{subfigure}[b]{0.45\textwidth}
    \includegraphics[width=\textwidth]{img/smooth-histogram-after}
    \caption{Obraz wyjściowy, poddany operacji wygładzenia histogramu}
    \label{fig:equalize_histogram_after}
  \end{subfigure}
  \caption{Operacja wygładzenia histogramu}\label{fig:histogram_smooth}
\end{figure}
