jeżeli} \; a \geqslant b
  \end{matrix}\right. 
\end{gather*}
\paragraph{Dodawanie obrazów z wagami} \mbox{}\\
Dodawanie obrazów z wagami zdefiniowane jest wzorem:
\begin{gather*}
  f^{(out)}(x, y) = \big\lceil \displaystyle\sum_{i=1}^{n} f^{(in)}_i(x, y) \cdot w_i\big\rceil, \quad (x, y) \in P^{(in)}_{N,M}.
\end{gather*}
\paragraph{Dodawanie obrazów wielokanałowych} \mbox{}\\
W przypadku obrazów wielokanałowych, operację dodawania wykonujemy dla każdego kanału osobno:
\begin{gather*}
  f^{(out)}(x, y, c) = \displaystyle\sum_{i=1}^{n} f^{(i)}(x, y), \quad (x, y, c) \in P^{(in)}_{N,M},
\end{gather*}
dla każdego
\end{document}
