\section{Rezultaty badań}
Badania przeprowadzone zostały na zbiorze zawierającym kilkaset obrazów przedstawiających tablice rejestracyjne wykonane w różnych warunkach. Moim głównym celem było porównywanie metod segmentacji obrazu, a metody rozpoznawania znaków traktowałem tylko jako metody pomocnicze. Dlatego kiedy liczba segmentów zwracanych przez algorytm zgadzała się z liczbą znaków oczekiwanego wyniku, a wartości wyników się nie zgadzały, osobiście sprawdzałem czy niepoprawne rozpoznanie tekstu zapisanego na tablicy rejestracyjnej jest wynikiem niepoprawnego działania algorytmu segmentacji obrazu, czy algorytmu rozpoznawania znaków. \\
Jako skuteczność metody w tym rozdziale będziemy rozumieć stosunek procentowy liczby poprawnie rozpoznanych przypadków testowych do liczby wszystkich przypadków testowych.

\subsection{Wpływ parametrów metod na rezultaty}
\subsubsection{Metoda wyszukiwania szczytów histogramu}
Zbadałem, jaki wpływ na rezultaty ma wyostrzenie obrazu przed faktycznym procesem segmentacji, oraz zmiana minimalnego dystansu pomiędzy szczytami histogramu. Wyniki przedstawione zostały w tabelach~\ref{tab:wyszukiwanie_szczytow_wyostrz} oraz~\ref{tab:wyszukiwanie_szczytow_dystans}. Jak można zauważyć, zastosowanie operacji wyostrzania obrazu powoduje prawie dwukrotny wzrost skuteczności algorytmu. \\
Również parametr minimalnej odległości pomiędzy szczytami znacząco wpływa na rezultaty. W przypadku braku minimalnej odległości (wartość 1), algorytm uzyskuje bardzo słabe rezultaty. Wynika to zapewne z faktu, iż często najwyższe ekstrema lokalne w histogramie występują bardzo blisko siebie, co powoduje, że próg wyznaczany jest w ramach tego samego obiektu, a nie pomiędzy dwoma różnymi obiektami. Zastosowanie większych minimalnych odległości przyniosło spodziewane rezultaty.

\begin {table}[H]
  \begin{center}
    \begin{tabular}{l | c}
      \space & Skuteczność \\
      \hline
      Segmentacja wyostrzonego obrazu & 49\% \\
      Brak wyostrzenia obrazu & 26\%
    \end{tabular}
    \caption {Wpływ użycia algorytmu wyostrzenia obrazu na wynik działania metody wyszukiwania szczytów histogramu}
    \label{tab:wyszukiwanie_szczytow_wyostrz} 
  \end{center}
\end {table}

\begin {table}[H]
  \begin{center}
    \begin{tabular}{c | c}
      Minimalna odległość od szczytów & Skuteczność \\
      \hline
      1 & 12\% \\
      25 & 28\% \\
      80 & 49\% \\
      150 & 49\%
    \end{tabular}
    \caption {Wpływ zmiany parametru minimalnej odległości między szczytami na wynik działania metody wyszukiwania szczytów histogramu}
    \label{tab:wyszukiwanie_szczytow_dystans} 
  \end{center}
\end {table}

\subsubsection{Metoda wyszukiwania minimum histogramu}
Parametrem tej metody była liczba sąsiednich punktów jakie zostały użyte do obliczenia nowej wartości dla danego punktu w histogramie (przy założeniu, że zostały zastosowane omawiane w poprzednim rozdziale ulepszenia). Tabela~\ref{tab:wyszukiwanie_minimum_histogramu_sasiady} przedstawia uzyskane przeze mnie wyniki. Można zauważyć, że wraz ze wzrostem liczby sąsiednich punktów, rosła skuteczność metody. Maksymalna wartość została osiągnięta dla liczby 56. Dla kolejnych wartości skuteczność spadła o kilka procent, jednak nadal była stosunkowo wysoka. Na podstawie wyników można więc stwierdzić, że użycie większej liczby sąsiadów w operacji wygładzania histogramu, zwiększa skuteczność algorytmu. \\
Jako kolejny parametr metody, przyjąłem użycie omawianych ulepszeń, polegających na wygładzeniu histogramu oraz zmiany sposobu wyszukiwania minimum. Zgodnie z moimi przewidywaniami, algorytm okazał się skuteczniejszy, w przypadku zastosowania moich poprawek. Różnica skuteczności metody bez ulepszeń, oraz po zastosowaniu poprawek, przedstawiona została w tabeli~\ref{tab:wyszukiwanie_minimum_histogramu_ulepszenia} \\
Nie wykonywałem porównania wpływu zastosowania operacji wyostrzania obrazu na jakość rezultatu (takie testy wykonywałem dla metody wyszukiwania szczytów histogramu), jednak prawdopodobnie wpływ niezastosowania tej operacji również negatywnie wpłynie na skuteczność metody.


\begin {table}[H]
  \begin{center}
    \begin{tabular}{c | c}
      \begin{tabular}[x]{@{}c@{}}Liczba sąsiadów użyta\\w algorytmie wygładzania histogramu\end{tabular} & Skuteczność \\
      \hline
      2 & 55\% \\
      8 & 56\% \\
      16 & 59\% \\
      28 & 60\% \\
      44 & 62\% \\
      56 & 66\% \\
      72 & 63\% \\
      88 & 63\%
    \end{tabular}
    \caption {Wpływ liczby sąsiednich punktów użytych do operacji wygładzania histogramu na działania metody wyszukiwania minimum histogramu}
    \label{tab:wyszukiwanie_minimum_histogramu_sasiady} 
  \end{center}
\end {table}


\begin {table}[H]
  \begin{center}
    \begin{tabular}{l | c}
      \space & Skuteczność \\
      \hline
      Segmentacja obrazu z ulepszeniami & 66\% \\
      Segmentacja obrazu bez zastosowania ulepszeń & 53\%
    \end{tabular}
    \caption {Wpływ zastosowania poprawek na działania metody wyszukiwania minimum histogramu}
    \label{tab:wyszukiwanie_minimum_histogramu_ulepszenia} 
  \end{center}
\end {table}

\subsubsection{Metoda Sauvola i Pietikainena}
Rezultaty badań przedstawione zostały w tabelach~\ref{tab:sauvol_pietikainen_dlugosc_boku} oraz~\ref{tab:sauvol_pietikainen_parametr_k}. Skuteczność przebadałem dla każdej kombinacji parametru $k$ oraz $S$. Jak można zauważyć w tabeli~\ref{tab:sauvol_pietikainen_parametr_k}, wartość parametru $k$ nie wpłynęła na zmianę skuteczności algorytmu dla problemu segmentacji tablic rejestracyjnych (dla przykładu zamieściłem w tabelach wyniki dla dwóch różnych wartości parametru $S$). Jak widać, niezależnie od wartości parametru $k$, skuteczność metody nie zmienia się dla stałego $S$. \\
Znaczący wpływ ma natomiast wartość parametru $S$. Jeśli parametr ten nie zostanie w ogóle uwzględniony ($S = 1$), skuteczność metody jest bardzo niska. Najlepszy wynik został natomiast osiągnięty dla wartości 10\% wysokości obrazu.


\begin {table}[H]
  \begin{center}
    \begin{tabular}{c | c | c}
      Wartość parametru $k$ & Skuteczność ($S = 10\%$) & Skuteczność ($S = 50\%$)  \\
      \hline
      0.05\% & 72\% & 53\% \\
      0.1\% & 72\% & 53\% \\
      0.2\% & 72\% & 53\% \\
      0.4\% & 72\% & 53\% \\
      0.5\% & 72\% & 53\% \\
      0.7\% & 72\% & 53\%
    \end{tabular}
    \caption {Wpływ zmiany wartości parametru k na skuteczność metody Sauvola i Pietikainena dla różnych wartości parametru $S$}
    \label{tab:sauvol_pietikainen_parametr_k} 
  \end{center}
\end {table}

\subsubsection{Metoda Sauvola i Pietikainena} 
\begin {table}[H]
  \begin{center}
    \begin{tabular}{c | c}
      Wartość parametru $S$ & Skuteczność \\
      \hline
      1\% & 5.4\% \\
      5\% & 63\% \\
      10\% & 72\% \\
      20\% & 64\% \\
      50\% & 53\% \\
      70\% & 54\% \\
      100\% & 50\%
    \end{tabular}
    \caption {Wpływ zmiany długości boku obszaru sąsiedztwa na skuteczność metody Sauvola i Pietikainena}
    \label{tab:sauvol_pietikainen_dlugosc_boku} 
  \end{center}
\end {table}


\subsection{Porównanie różnych metod segmentacji obrazów}
Tabela~\ref{tab:all_methods_comparision} przedstawia porównanie wszystkich testowanych przeze mnie metod. Algorytmy uruchamiane były z takimi parametrami, dla jakich zwracały najlepsze wyniki.

\begin {table}[H]
  \begin{center}
    \begin{tabular}{l | c}
      Metoda & Skuteczność \\
      \hline
      Wyszukiwanie szczytów histogramu & 49\% \\
      Wyszukiwanie minimum histogramu & 66\% \\
      Metoda Sauvola i Pietikainena & 72\%
    \end{tabular}
    \caption {Wyniki działania metod segmentacji obrazu na przykładowych danych testowych}
    \label{tab:all_methods_comparision} 
  \end{center}
\end {table}

\subsection{Analiza błędów segmentacji tablic rejestracyjnych}
Jednym z częstszych błędów segmentacji były złączenia znaku z ramką tablicy rejestracyjnej. Powodowało to sklasyfikowanie danego znaku oraz ramki jako jednego obiektu, który nie odpowiadał np. kryterium stosunku wysokości do szerokości znaku. Przykład niepoprawnie wykonanej segmentacji obrazu spowodowanej złączeniem się znaku z ramką przedstawiony został na rysunku~\ref{fig:polaczone_krawedzie}. Widzimy, że litera \textit{P} oraz \textit{I} zostały połączone wraz z ciemnym obszarem u dołu obrazu, przez co nie mogły zostać sklasyfikowane jako znaki należące do tablicy rejestracyjnej. \\
Innym podobnym problemem było występowanie znaków ze sobą połączonych. Na tablicach rejestracyjnych mógł pojawić się dodatkowy element, który znajdował się pomiędzy dwoma znakami. W takiej sytuacji dwa sąsiednie znaki sklasyfikowane zostały jako jeden obiekt. 
\begin{figure}
  \centering
  \includegraphics[width=0.5\textwidth]{img/polaczone-krawedzie}
  \caption{Przypadek niepoprawnego odrzucenia znaków - znaki połączone z obszarem nienależącym do tablicy rejestracyjnej}
  \label{fig:polaczone_krawedzie}
\end{figure}

