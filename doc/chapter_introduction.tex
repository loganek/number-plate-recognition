\section{Wstęp}
Analiza obrazu odgrywa coraz większą rolę w~szeroko pojętej inspekcji obiektów. Dzięki algorytmom wizji komputerowej systemy informatyczne są coraz częściej wykorzystywane w~przemyśle, w~miejscach gdzie do tej pory pracę musiał wykonywać człowiek. Systemy oparte o~analizę wizji komputerowej mogą pracować o~wiele wydajniej oraz popełniają o~wiele mniej błędów. Ponadto są tanie w~utrzymaniu. Standardem jest już na przykład inspekcja wizyjna jakości na taśmach produkcyjnych. Przykładem wykorzystania wizji komputerowej w~rozwiązaniach przemysłowych może być weryfikacja butelki oddanej do automatu przyjmującego butelki zwrotne. System na podstawie zdjęcia może stwierdzić, czy kształt butelki jest akceptowany w~danej maszynie oraz czy butelka nie jest uszkodzona, i~w zależności od rezultatu oceny, wypłacić użytkowi pieniądze lub zwrócić butelkę.
\paragraph{}
Segmentacja odgrywa ważną rolę w~analizie obrazu, ponieważ pozwala na wyodrębnienie z~obrazu cech, które są istotne z~punktu widzenia danego problemu. Na obrazie, na którym zarejestrowany jest badany obiekt, znajdują się bardzo często inne obiekty, które są zbędne z~punktu widzenia rozwiązania danego problemu, a~nawet przeszkadzają w~jego rozwiązaniu. Zadaniem segmentacji jest wydobycie z~obrazu tylko takich obiektów, które mogą być przydatne w~kolejnym etapie, jakim jest analiza obiektu.
\paragraph{}
Metody segmentacji są jednym ze sposobów rozpoznawania ciągów znaków. Dzięki metodom segmentacji możemy z~pośród całego tekstu wyodrębnić zbiór pojedynczych znaków, a~następnie przekazać je do algorytmu rozpoznawania pojedynczych znaków.
\paragraph{}
Celem mojej pracy jest zaprezentowanie podstawowych algorytmów przetwarzania wizji komputerowej oraz metod segmentacji obrazów. Metody segmentacji omówię pod kątem rozwiązywania problemu segmentacji tekstu, a~w szczególnym przypadku - segmentacji obrazów tablic rejestracyjnych pojazdów. Przedstawię problemy, jakie wiążą się z~zagadnieniem segmentacji obrazów tablic rejestracyjnych, oraz omówię zaproponowane przeze mnie rozwiązania tych problemów. Głównym celem jest porównanie pod względem jakościowym algorytmów segmentacji obrazu w~zastosowaniu segmentacji obrazów tablic rejestracyjnych pojazdów. Dlatego omówię algorytmy, jakie zdecydowałem się wykorzystać do przeprowadzenia badań. Następnie przedstawię sposób, w~jaki przeprowadzałem badania, omówię środowisko, jakie przygotowałem w~celu przeprowadzenia testów algorytmów. Na końcu mojej pracy zaprezentuję uzyskane przeze mnie wyniki.
\paragraph{}
Na początku pracy opisałem podstawowe operacje wykonywane na obrazach, wprowadziłem pojęcia, których będę używał w~dalszych rozdziałach. Następnie przedstawiłem problem segmentacji obrazu oraz zaprezentowałem metody służące do segmentacji obrazu. W~dalszej części pracy przedstawiłem algorytmy rozpoznawania znaków. Następnie wymieniłem problemy związane z~segmentacją obrazów tablic rejestracyjnych, a~także zaproponowałem metody rozwiązywania tych problemów. W~kolejnym rozdziale przedstawiłem proces badania metod segmentacji obrazów tablic rejestracyjnych, opisałem algorytmy poddane badaniom. Na końcu pracy zaprezentowałem i~porównałem wyniki moich badań, a~także podsumowałem moją pracę.
