\documentclass[12pt,a4paper,notitlepage,twoside]{article}
\usepackage{polski}
\usepackage[utf8]{inputenc}
\usepackage{mathtools}
\usepackage{helvet}
\usepackage[top=2.5cm, bottom=2.5cm, left=3cm, right=2.5cm]{geometry}
\usepackage{graphicx}
\usepackage[absolute]{textpos}
\usepackage{tocloft}
\usepackage{cite}
\usepackage[numbib,notlof,notlot,nottoc]{tocbibind}
\usepackage{indentfirst}
\usepackage{fancyvrb}
\usepackage{listings}
\usepackage{float}
\linespread{1.3}

\renewcommand{\cftsecleader}{\cftdotfill{\cftdotsep}}
\renewcommand\figurename{Rys.}
%%\setlength\cftparskip{-2pt}
\setlength\cftbeforesecskip{1pt}
\setlength\cftaftertoctitleskip{2pt}

\lstset{
  numbers=left,
  breaklines=true,
  captionpos=b,
  language=C++,
  xleftmargin=\parindent,
  basicstyle=\small,
  inputencoding=utf8,
  escapeinside={\%*}{*)}
}

\usepackage{hyperref}
\hypersetup{
  colorlinks=true, %set true if you want colored links
  linktoc=all,     %set to all if you want both sections and subsections linked
  linkcolor=black,  %choose some color if you want links to stand out
  citecolor=black,
  urlcolor=black,
}

\setlength{\TPHorizModule}{10mm}
\setlength{\TPVertModule}{\TPHorizModule}
\textblockorigin{30mm}{25mm}

\makeatletter

\renewcommand{\maketitle}{\begin{titlepage}

    \begin{center}
      \includegraphics[width=2.5cm]{img/polsl-logo}
    \end{center}

    \vspace{0.5cm}
    \begin{center}
      \Large{\textbf{\textsc{Politechnika Śląska\\
	    Wydział Automatyki, Elektroniki i Informatyki\\
	    Kierunek Informatyka\\}}}
    \end{center}

    \vspace{0.5cm}
    \begin{center}
      \Large{Projekt inżynierski}
    \end{center}

    \begin{center}

      \vspace{0.5cm}
	     {\fontsize{30}{36}\selectfont \@title}

    \end{center}

    \begin{textblock}{10}(0,19)
      \noindent{\fontsize{14}{21}{Autor: \@author\\
	  Kierujący pracą: dr inż. Zbigniew Szaszkowski}}
    \end{textblock}
    \begin{textblock}{15.5}(0,24)
      \begin{center}
	Gliwice, czerwiec 2015
      \end{center}
    \end{textblock}
  \end{titlepage}

}

\makeatother

\author{Marcin Kolny}

\title{Rozpoznawanie ciągów znaków z wykorzystaniem metod segmentacji obrazu}

\begin{document}

\maketitle
\newpage
\tableofcontents
\newpage
\section{Podstawowe informacje dotyczące przetwarzania obrazów}
\subsection{Podstawowe definicje}
\subsubsection{Piksel}
Piksel jest najmniejszym elementem obrazu cyfrowego. Piksel może określać
\begin{itemize}
  \item odcień szarości
  \item kolor, wtedy \(f(x, y) = [a(x, y), b(x, y),...]\)\\
    gdzie $a, b,...$ to natężenie poszczególnych barw
  \item wskaźnik na element tablicy barw
\end{itemize}
\subsubsection{Obraz cyfrowy}
Obraz cyfrowy jest macierzą pikseli, postaci
\begin{gather*}
  f = f(x, y), x = 0,1,2,...,N-1; y = 0,1,2,...,M-1
\end{gather*}
gdzie
\begin{itemize}
  \item \(f(x, y)\) - pojedynelement macierzy, piksel
  \item \(M, N\) - szerokość oraz wysokość obrazu
\end{itemize}
\subsubsection{Obraz binarny}
Obraz binarny jest to obraz cyfrowy którego składowe piksele przyjmują wartości
\begin{gather*}
  f(x, y) = z, z \in {0, 1}
\end{gather*}.
\newpage
\cite{foo}

\bibliographystyle{plain}
\bibliography{main_document}

\end{document}
