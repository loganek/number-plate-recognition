\section{Podsumowanie i wnioski}
Głównym celem mojej pracy było przeprowadzenie badań dotyczących jakości różnych metod segmentacji obrazu, bazując na zbiorze testowym zawierającym tablice rejestracyjne pojazdu. Specyficzny zbiór testowy dostarczył mi kilku problemów, dla których zaproponowałem rozwiązania, a także wprowadziłem je w życie dostarczając ich implementację w swojej aplikacji.
\paragraph{}
Oprócz wykonanych porównań pomiędzy różnymi metodami segmentacji tekstu, sprawdziłem też wpływ zmiany parametrów pojedynczych metod na ich skuteczność. Jak zostało to pokazane w tabelach, bardzo często zmiana parametru algorytmu może mieć duży wpływ na jego końcową skuteczność.
\paragraph{}
Niestety, żaden z przetestowanych przeze mnie algorytmów, nie uzyskał skuteczności bliskiej 100\%. Istnieje wiele problemów, dla których bardzo ciężko znaleźć skuteczne oraz uniwersalne rozwiązanie. Pomimo licznych problemów, jakie udało mi się rozwiązać, pozostało jeszcze wiele zagadnień które trzeba rozważyć, oczekując od algorytmu stu procentowej dokładności. Niestety wzięcie pod uwagę wszystkich sytuacji jest niemalże niemożliwe.
\paragraph{}
Uważam, że bardzo trudne jest uzyskanie pewności wyniku pomiaru, bazując tylko i wyłącznie na obrazie tablicy rejestracyjnej. Mimo to uważam, że wyniki, jakie udało mi się osiągnąć podczas badań były zadowalające, i spełniły moje założenia sprzed przystąpienia do realizacji pracy.
